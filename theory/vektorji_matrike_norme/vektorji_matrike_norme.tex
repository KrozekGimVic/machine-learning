\section{Vektorji, Matrike in Norme} 
\subsection{Vektorski prostor}
\paragraph{Definicija:} Vektorski prostor je četverica ($V, O, +, \cdot$), kjer je:\\
\indent $V \dots$ množica vektorjev\\
\indent $O \dots$ obseg skalarjev\\
\indent $+ \dots$ dvomestna operacija $+: V \times V \rightarrow V$\\
\indent $\cdot \dots$ produkt s skalarjem $\cdot: O \times V \rightarrow V$\\

\paragraph{Kjer:}\ \\
\indent $\forall x, y, z$\indent 
\begin{minipage}{0.5\textwidth}
	$x + (y + z) = (x + y) + z$\\
	$x + y = y + x$
\end{minipage}\\\\
\indent $\exists 0 \in V:\ x+0 = x$\\
\indent $\exists 1 \in O:\ 1\cdot x = x$\\
\indent $\forall x\ \exists y:\ x + y = 0$\\
\indent $\forall x \in V\ \forall \mu, \lambda \in O:\ \lambda \cdot (\mu \cdot x) = (\lambda \cdot \mu) \cdot x$\\
\indent $\forall x,y\ \forall \lambda :\ \lambda \cdot (x + y) = \lambda \cdot x + \lambda \cdot y$\\
\indent $\forall x\ \forall \lambda, \mu :\ (\lambda + \mu) \cdot x = \lambda \cdot x + \mu \cdot x$\\

\noindent $V$ je vektorski prostor nad $O$.
\paragraph{Primeri:}\ \\
\indent $(\R, \R, +, \cdot)$\\
\indent $(\R^3, \R, +, \cdot)$\\
\indent $(\R^{10\times 10}, \R, +, \cdot)$\\
\indent $(P_{\leq 7}, \mathbb{C}, +, \cdot)$\\
\indent $(\mathbb{C}[x], \mathbb{C}, \cdot, \cdot)$\\
\indent $(\R[x], \mathbb{C}, +, \cdot) // \indent (5x + 4)i \rightarrow 5xi + 4i \notin \R[x]$\\
\indent $(\R \rightarrow \R, \R, +, \cdot)$

\subsection{Skalarni produkt}
\paragraph{Definicija:} Skalarni produkt nad realnim vektorskim prostorom $(V, \R, +, \cdot)$ je preslikava $:<\cdot , \cdot >: V\times V \rightarrow \R$ in velja:\\
\indent $\forall x, y:\ <x, y> = <y, x>$\\
\indent $\forall x,y,z:\ <x, y+z> = <x,y> + <x, z>$\\
\indent $\forall x:\ <x, x> \geq 0$\\
\indent $\forall x:\ <x, x> = 0 \Rightarrow x=0$\\
\indent $<x, y> = 0 \Rightarrow$ pravokotna

\paragraph{Primer:} najdi skalarni produkt nad $\R^3$\\
\indent $<(a_1, a_2, a_3), (b_1, b_2, b_3)> = a_1b_1 + a_2b_2 + a_3b_3$\\
\indent $<v, u> = \sum_{i=1}^{n}v_iu_i$


\subsection{Transponiranje}
$(A^T)_{ij} = a_{ji}$

\subsection{Sled} 
\paragraph{def:} $sl(A) = \sum a_{ii}$\\
\indent $<A, B> = sl(A^TB)$\\
\indent $sl(A^T) = sl(A)$\\
\indent $sl(AB) = sl(BA)$

\paragraph{Dokaz:}
\[ sl(AB) = \sum_{i}(AB)_{ii} = \sum_i \sum_k a_{ik}b_{ki} = \sum_k \sum_i b_{ki}a_{ik} = \sum_k (BA)_{kk} = sl(BA)\]

\subsection{Norme}
\paragraph{Definicija:} Norma na realnem prostoru $V$ je preslikava $||\cdot||:\ V \rightarrow \R$ za katero velja:\\
\indent $\forall x \in V:\ ||x|| \geq 0$\\
\indent $\forall x \in V:\ ||x|| = 0 \Rightarrow x=0$\\
\indent $||x + y|| \leq ||x|| + || y || $\\
\indent $\forall \lambda \in V:\ ||\lambda \cdot x|| = |\lambda| \cdot ||x||$

\paragraph{Primeri:}\ \\
\indent $||x||_2 = \sqrt{<x, x>} = \sqrt{\sum x_i^2}$\\
\indent $||x||_1 = \sum |x_i|$\\
\indent $||x||_p = \sqrt[p]{\sum x_i^p}$\\
\indent $||x||_\infty = \lim_{p\rightarrow \infty}||x||_p = \max \{|x_i|\}$

\paragraph{Dokaz:}
\[\begin{split} \lim_{p\rightarrow \infty} \sqrt[p]{x_1^p + \dots + x_n^p} = 
\lim_{p\rightarrow \infty} \sqrt[p]{\frac{x_1^p + \dots + x_n^p}{\max\{|x_i|\}^p}}  \cdot \max \{|x_i|\} =\\
=\max\{|x_i|\}\cdot \lim_{p\rightarrow \infty} \sqrt[p]{(\frac{x_1}{m})^p + \dots + (\frac{x_n}{m})^p} = \max \{|x_i|\}\end{split}\]





















